\chapter Wprowadzenie

\TODO biometryk nie da się zgubić choć część populacji jest niema

\section{Weryfikacja mówcy}

Weryfikacja mówcy (\textit{speaker verification}).

Pojęcie weryfikacji mówcy jest związane z szerszym zagadnieniem rozpoznawania mówcy (\textit{speaker recognition}). Jest to ogólna nazwa stosowana dla procedur wykorzystujących informację biometryczną zawartą w sygnale mowy. Kluczowe jest odróżnienie procesu rozpoznawania mówcy od systemów rozpoznawania mowy (\textit{speech recognition}). W drugim wymienionym problemie nie jest istotna tożsamość osoby wypowiadającej się, a jedynie sens jej wypowiedzi. Zatem cała informacja biometryczna zarejestrowanego sygnału mowy jest niewykorzystywana. Z drugiej jednak strony, w systemach rozpoznawania mówcy wykorzystuje się niejednokrotnie techniki rozpoznawanie treści języka naturalnego m. in. w odmianach omawianych systemów opartych na nagromadzonej wiedzy (\textit{knowledge-based systems}), których zadaniem jest wzmocnienie procesu weryfikacji oraz zapobieganiu wystąpienie problemu żywotności (\textit{liveness issue}). 

\TODO opis modelu identyfikacji/weryfikacji mówcy ?

\TODO na co składają się charakterystyki biometryczne sygnału mowy ? vocal tract ?

\textit{text-dependent speaker recognition} \cite{fosr}

Problem żywotności polega na możliwości oszukania działającego systemu weryfikacji mówcy poprzez dostarczenie na wejście takiego systemu spreparowany sygnał mowy - na przykład wysokiej jakości nagranie weryfikowanego mówcy, edytowane w odpowiedni sposób. Wraz z rozwojem technik audio zmylenie systemu niezabezpieczonego ze względu na ten typ ataku staje się coraz łatwiejsze.

\textit{text-independent speaker recognition} \cite{fosr}

\textit{text-prompted speaker recognition} \cite{fosr}

Implementacja systemu weryfikacji mówcy jest nazywana automatycznym systemem rozpoznawania mówcy (\textit{automatic speaker verification system).

\TODO zastosowania 

\TODO system weryfikacji mówcy może być skojarzony z innymi systemami rozpoznawania biometryk

\TODO opis zastosowania weryfikacji mówcy w systemie wbudowanym


