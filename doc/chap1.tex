\chapter Wprowadzenie

\%TODO
biometryk nie da się zgubić choć część populacji jest niema. Istniejąca infrastruktura pomaga. Łatwość uzyskania sygnału mowy, nawet bez kooperacji.

\section{Weryfikacja mówcy}

Weryfikacja mówcy (\textit{speaker verification}).

Proces weryfikacji mówcy jest związany z szerszym zagadnieniem - rozpoznawania mówcy (\textit{speaker recognition}), które charakteryzuje ogół metod wykorzystujących dane biometryczne zawarte w sygnale mowy w celu określenia tożsamości.

Sygnał mowy może być rozpatrywany jako cecha biometryczna. Sygnał mowy charakteryzowany jest przez budowę aparatu głosowego człowieka, która jest mniej lub bardziej unikatowa dla każdego człowieka, umożliwiając rozróżnienie badanej jednostki na tle populacji.

Ogólną strukturę problemu rozpoznawania mówcy można rozłożyć na na trzy elementy.\cite{fosr} Po pierwsze, konieczne jest aby tworzony system dysponował modelem charakterystyk aparatu głosowego człowieka. Model taki dla przykładu może przybrać formę modelu fizyko-matematycznego aparatu głosowego człowieka. Otrzymany model musi umożliwiać parametryzację - skojarzenie z konkretną osobą. Model taki tworzony jest poprzez analizę sygnału mowy. Dopiero na tej podstawie możliwe jest porównywanie modelu utworzonego przy użyciu testowanego sygnału z modelem odniesienia. Forma i cel tego porównania definiują podklasę problemu rozpoznawania mówcy.

Weryfikacja mówcy charakteryzuje się wykonaniem dwóch kluczowych porównań - pierwszego pomiędzy modelem utworzonym z poddanego weryfikacji sygnału mowy a pamiętanym modelem osoby której dotyczy weryfikacja. W odróżnieniu of problemu identyfikacji, podczas weryfikacji mówcy potrzebna jest więc znajomość tożsamości osoby poddanej weryfikacji.
Drugie z kolei porównanie dokonywane jest pomiędzy modelem poddanym weryfikacji, a uogólnionym modelem całej populacji (\textit{background model}) lub pewnej jej podgrupy (\textit{cohort model}. Na podstawie relacji tych dwóch odległość podejmowana jest decyzja o autoryzacji.

W przypadku kiedy nie jest możliwa lub pożądana znajomość przez system tożsamości osoby weryfikowanej przed dokonaniem autoryzacji, możliwe jest zastosowanie bardziej złożonego problem identyfikacji mówcy na otwartym zbiorze.(\textit{open-set speaker identification}). Proces ten można uważać jako złożenie problemu weryfikacji mówcy oraz identyfikacji mówcy na zbiorze zamkniętym (\textit{close-set speaker identification}). Polega on na przeprowadzeniu weryfikacji mówcy na modelu uzyskanym z procesu identyfikacji mówcy na zbiorze zamkniętym, która dokonuje porównania z całą dostępną bazą modeli mówców i zwraca ten najbliższy modelowi testowanemu. Problem taki jest więc obliczeniowo co najmniej tak złożony jak weryfikacja mówcy (dla bazy w której znajduje się tylko jeden mówca).

\section{relacja pomiędzy rozpoznawanie mowy, a rozpoznawaniem mówcy}
Kluczowe jest odróżnienie procesu rozpoznawania mówcy od systemów rozpoznawania mowy (\textit{speech recognition}). Pomiędzy tymi dwoma rozpatrywanymi dziedzinami z zakresu analizy sygnału mowy występuje dychotomiczny podział. Wynika to z tego, że sygnał mowy jest sygnałem bogatym informacyjnie oraz że jedynie mała część tej informacji posiada znaczenie semantyczne, zaś reszta niesie wiedzę o budowie konkretnego, ludzkiego narządu mowy. W problemie rozpoznawania mowy nie jest istotna tożsamość osoby wypowiadającej się, a jedynie sens jej wypowiedzi. Zatem reszta sygnału nie zawierająca odczytywanej wiadomości jest redundantna z punktu widzenia tego zagadnienia - cała informacja biometryczna jest niewykorzystywana, co za tym idzie często filtrowana przez zaimplementowany system. Z drugiej strony, w systemach rozpoznawania mówcy, w samym sednie jego zainteresowania, abstrahuje się od treści mowy. Stanowi ona jedynie środek dla dostarczenia informacji o fizjologii aparatu mowy. Dlatego prawdopodobnie system rozpoznawania mówcy usunie treść mowy, a utworzy jedynie model aparatu głosowego. Usprawiedliwia to twierdzenie o rozłączności tych dziedzin ze względu na zainteresowanie informacją zawartą w sygnale mowy. 

Okazuje się, że wspomniana wyżej zależność powoduje to, że techniki przetwarzania sygnału stosowane przy analizie obu dziedzin są w zasadzie bardzo podobne. 

W przypadku rozpoznawania mówcy zależnego od wypowiadanego tekstu czy rozpoznawania mówcy z generowanym tekstem informacja semantyczna wykorzystywana jest jedynie do określenia zakresu badanych głosek czy zapobieganiu problemowi żywotności. Informacja ta zatem nie wpływa na postać stosowanych technik rozpoznawania mówcy, a jedynie na optymalny ich dobór - ujawnia kontekst użycia. Innym przykładem tego typu jest zastosowania technik rozpoznawania treści języka naturalnego m. in. w odmianach omawianych systemów opartych na nagromadzonej wiedzy (\textit{knowledge-based systems}), których zadaniem jest jedynie wzmocnienie procesu weryfikacji oraz zapobieganie wystąpienia problemu żywotności (\textit{liveness issue}). 

\section{Klasyfikacja weryfikacji mówcy ...?}
\textit{text-dependent speaker recognition} \cite{fosr}

Problem żywotności polega na możliwości oszukania działającego systemu weryfikacji mówcy poprzez dostarczenie na wejście takiego systemu spreparowany sygnał mowy - na przykład wysokiej jakości nagranie weryfikowanego mówcy, edytowane w odpowiedni sposób. Wraz z rozwojem technik audio zmylenie systemu niezabezpieczonego ze względu na ten typ ataku staje się coraz łatwiejsze.

\textit{text-independent speaker recognition} \cite{fosr}

\textit{text-prompted speaker recognition} \cite{fosr}

Implementacja systemu weryfikacji mówcy jest nazywana automatycznym systemem rozpoznawania mówcy (\textit{automatic speaker verification system}).

\%TODO zastosowania 

\%TODO system weryfikacji mówcy może być skojarzony z innymi systemami rozpoznawania biometryk

\%TODO opis zastosowania weryfikacji mówcy w systemie wbudowanym

\section{SYGNAŁ MOWY}

Człowiek dysponuje doskonałymi narzędziami do przeprowadzenia procesów rozpoznawania mowy oraz rozpoznawania mówcy. Analiza oraz zrozumienie mechanizmów powstawania mowy u człowieka dostarcza podstaw do sformułowania metod syntezy języka naturalnego. Podobna analiza systemu percepcji mowy na który składa się aparat słuchowy oraz układ nerwowy związany z dekodowaniem sygnału mowy daje podstawy do identyfikacji cech którymi posługuje się ludzki organizm do efektywnego rozpoznawania mówcy.

Układ produkcji mowy oraz jej percepcji są ze sobą nierozerwalnie związane. Sposób ekstrakcji informacji przez narząd słuchu odzwierciedla fizjologię produkcji mowy - zatem może wskazać najważniejsze cechu sygnału mowy dla naturalnego procesu rozpoznania mówcy. Dlatego wydaje się właściwe prześledzenie związków pomiędzy tymi dwoma elementami.

%fosr 5.0
Ludzki układ percepcji dokonuje rozróżnienia sygnałów audio poprzez rozróżnienie trzech własności: wysokości dźwięku, głośności oraz barwy dźwięku - tembru. 
%
\subsection{Produkcja sygnału mowy.}

%TODO
\subsubsection{Aparat mowy człowieka.}
Tembr głosu mówcy ustalony jest przez budowę jego dróg głosowych.

\subsection{Percepcja sygnału mowy przez człowieka}

\subsubsection{Aparat słuchowy.}
Narząd słuchu człowieka można rozpatrywać jako transduktor, co znaczy, że mapuje zmiany ciśnienie akustycznego w powietrzu na sygnał elektryczny w układzie nerwowym. Na samym początku toru przetwarzania sygnału audio znajduje się małżowina uszna, której zadaniem jest skupienie dźwięku. Sygnał akustyczny wpadający do kanału słuchowego jest filtrowany ze względu na jego fizyczne rozmiary, usuwane są niskie częstotliwości. Zmiany ciśnienia akustycznego zamieniane są na fale mechaniczne w ciele stałym na błonie bębenkowej, a następnie wzmacniane przez układ kosteczek słuchowych - młoteczka, kowadełka i strzemiączka. Strzemiączko łączy się z uchem wewnętrznym poprzez błonę okienka owalnego(łac. \textit{fenestra vestibuli}), którego zadaniem jest wzbudzenie drgań %TODO w domu zobacz kurwa anatomie ucha xD     

%TODO
\subsubsection{Układ nerwowy}

\subsection{Lingwistyka a rozpoznawania mówcy.}

Występuje silny związek pomiędzy językiem a procesem rozpoznawania mówcy. Obszarami lingwistyki, które szczególnie dotyczą badanej kwestii są: fonetyka, fonologia oraz prozodia.

% fonetyka stan 0/10, plagiat z fosr
\subsubsection{Fonetyka}
Fonetyka zajmuje się badaniem dźwięków produkowanych przez aparat mowy człowieka. Elementarnym dźwiękiem rozpatrywanym przez fonetykę jest głoska. Z punktu widzenia całej lingwistyki jest to najmniejszy segment mowy. Budowa i funkcjonalność narządu mowy determinują zakres produkowanych głosek. Fonetyka bada podstawowe dźwięki mowy bez rozróżnianie ze względu na konkretny język czy znaczenie głoski.
Fonem jest najmniejszą jednostką mowy na podstawie której możliwa jest interpretacja jej znaczenia. To znaczy, że jest semantycznie istotna. Fonem rozpatruje się ze względu na znaczenie w konkretnym języku. Głoska może być realizacją fonemu. Dwie różne głoski mogą stanowić realizację tego samego fonemu - to znaczy nieść tę samą informację semantycznie. Dla przykładu 
%TODO przyklad
Zbiór takich głosek nazywany jest alofonem.

Z punktu widzenia mechanizmów powstawania dźwięków w ludzkim narządzie mowy, można wyróżnić trzy z których zbudowany jest każdy emitowany dźwięk mowy. Składa się na nie:
- dźwięk rezonujący powstały w wibrującym źródle (np. drgające fałdy głosowe) i rezonujący w przestrzeni rezonansowej na którą składają się drogi oddechowe znajdującą się powyżej krtani,
- dźwięk powstały przez nielaminarny przepływ powietrza,
- dźwięk impulsowy powstały przez energiczne wypuszczenie powietrza z układu oddechowego.

%TODO fonologia
\subsubsection{Fonologia}
Fonologia bada

%plagiat z wikipedii xd 
\subsubsection{Prozodia}
Prozodia zajmuje się brzmieniowymi właściwościami mowy na które składają się trzy elementy - intonacja, akcent oraz iloczas.


