\documentclass[
  printmode
]{mgr}

%encoding
%poni�ej deklaracje u�ycia pakiet�w, usun�� to co jest niepotrzebne
\usepackage{polski}       %przydatne podczas sk�adania dokument�w w
%%j. polskim 
%\usepackage[polish]{babel} %alternatywnie do pakietu
%%polski, wybra� jeden z nich
\usepackage[latin2]{inputenc} %kodowanie znak�w, zale�ne od systemu
\usepackage[T1]{fontenc}

%graphics
\usepackage{graphicx}
\usepackage{subfigure}
\usepackage{psfrag}

%math
\usepackage{amsmath}
\usepackage{amsfonts}

%tables
\usepackage{supertabular}
\usepackage{array}
\usepackage{tabularx}
\usepackage{hhline}

% ??
\usepackage{showlabels}

%dane do z�o�enia strony tytu�owej
\title{Tytu� pracy magisterskiej}
\engtitle{Master thesis title}
\author{Jan Nowak}
\supervisor{dr hab. in�. Imi� Nazwisko Prof. PWr, I-6}
\field{Automatyka i Robotyka (AIR)}
\specialisation{Robotyka (ARR)}

\begin{document}
\bibliographystyle{plabbrv} %tylko gdy u�ywamy BibTeXa, ustawia polski
                            %styl bibliografii

\maketitle %polecenie generuj�ce stron� tytu�ow� 


\tableofcontents %spis tre�ci


\addcontentsline{toc}{chapter}{Bibliografia} %utworzenie w
                                             %spisietre�ci pozycji
                                             %Bibliografia

\bibliography{bibliografia} % wstawia bibliografi� korzystaj�c z pliku
                            % bibliografia.bib - dotyczy BibTeXa,
                            % je�eli nie korzystamy z BibTeXa nale�y
                            % u�y� otoczenia thebibliography

%opcjonalnie mo�e si� tu pojawi� spis rysunk�w i tabel
% \listoffigures
% \listoftables
\end{document}

