\chapter{Podsumowanie}

Powstały w niniejszej pracy system weryfikacji mówcy osiągnął dobry wynik błędu ERR równy 6\% w teście zawierającym ograniczony materiał nagrań. Każde nagranie zostało zarejestrowane w podobnych warunkach z podobnymi zakłóceniami tła oraz zawierało tę samą treść (wyświetlane hasło). Z tych powodów aby otrzymać bardziej wiarygodne wyniki efektywności otrzymanego systemu należy wykonać bardziej rozległe testy, najlepiej przy użyciu dużych baz wypowiedzi mówców powstałych w tym właśnie celu takie jak baza NIST (\textit{National Institute of Standards and Technology}) albo ELRA (European Language Resources Association). Implementacja otrzymanego przykładowego systemu składa się tylko z niezbędnych elementów potrzebnych do jego uruchomienia. Jednak przedstawiona w rozdziale 2 architektura systemu pozwala na szeroką ingerencję jeżeli chodzi o dodawanie do systemu nowych technik poprawiających efektywność systemu. W uruchomionym na potrzeby tej pracy systemie pożądanym ulepszeniem jest wprowadzenie technik detekcji mówcy (VAD) tak aby puste ramki cech MFCC (pomiędzy słowami) nie wpływały na efekt procesu tworzenia modelu mówcy (ze względu na faworyzowanie modeli z centroidami bliżej punktu zerowego). Tak jak wspomniano na wstępie pracy, warto dostarczyć do systemu więcej materiału trenującego (np. parokrotne powtórzenia wszystkich cyfr dla każdego mówcy w przypadku systemu z wyświetlanym hasłem). Kolejnym ulepszeniem systemu mogłoby być w zastosowaniach z przetwarzaniem sygnału mowy w czasie rzeczywistym uwzględnienie podsystemu \textit{speaker pruning} (opisanego w sekcji \textbf{\ref{speakerprunning}}) może to okazać się przydatne dla zwiększenia szybkości pracy systemu dla weryfikacji z modelami kohort.

Autor uważa, że zaproponowana architektura i sposób implementacji w języku C++ może być bazą do stworzenia obszernego narzędzia dla prac związanych z konstruowaniem systemów weryfikacji mówcy dla systemów wbudowanych w czasie rzeczywistym. Wydaje się, że jest możliwa efektywna i jasna implementacja w proponowanym środowisku większości technik weryfikacji mówcy omawianych w części teoretycznej tej pracy.
