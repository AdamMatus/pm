\chapter{System weryfikacji mówcy}

W tym rozdziale zostanie opisana implementacja pełnego systemu weryfikacji mówcy. 

\section{Platforma uruchomieniowa}

Weryfikacja poprawności działania systemu i wstępna implementacja zostały wykonane przy pomocy środowiska obliczeniowego Matlab. Rozwijanie oprogramowania w języku C++ na bazie narzędzi GNU odbywało się na systemie operacyjnym linux w dystrybucji debianowej. Docelową platformą uruchomieniową był komputer jednopłytkowy również z systemem takiego typu.

\section{Wybór technologii}

Omawiany system wykonuje parametryzacji sygnału mowy przy pomocy techniki Cepstralnych Współczynników Melowych (MFCC) w sekcji \textbf{\ref{mfcc}}. Rozpoznanie wzorca jest realizowane przy pomocy techniki kwantyzacji wektorów (VQ) (\textbf{sekcja \ref{vq}}), zaś proces modelowania mówcy dodatkowo korzysta z algorytmu LGB (\textbf{sekcja \ref{lgb}}).

\subsection{Sposób implementacji algorytmów}

Do implementacji algorytmów skorzystano ze znajdującej się w standardzie C++ bibliotek STL (Standard Template Library). W omawianej implementacji wykorzystywane są głównie trzy funkcje pochodzące z biblioteki STL są to:
\begin{itemize}
  \item{\textbf{std::generate(container\_iter start, container\_iter end, object\_func lambda)}} - funkcja generuje wartość dla każdego elementu pomiędzy iteratorem \textit{start} a iteratorem \textit{end} w jednym kontenerze za pomocą obiektu funkcyjnego \textit{lambda}.
  \item{\textbf{std::transform(container1\_iter start1, container1\_iter end1,
                       container2\_iter start2,
                       container3\_iter start3, object\_func lambda)}} 
  - funkcja generuje wartości w kontenerze o początku w \textit{start3} za pomocą obiektu funkcyjnego \textit{lambda} przyjmującego za argumenty kolejno wartości od start1 oraz od start2 aż do iteratora end1 i odpowiedniego w kontenerze nr. 2.
  \item{\textbf{std::for\_each(container\_iter start, container\_iter end, object\_func lambda)}} - produkuje wartości w kontenerze pomiędzy iteratorami start i end za pomocą obiektu funkcyjnego \textit{lambda} przyjmującego za argument wartości z tego kontenera. Funkcja może być zastąpiona pętlą zakresową: \textbf{for(auto \&\&x: container)}.

\end{itemize}

\section{Implementacja algorytmów wstępnego przetwarzania  sygnału mowy}

W tym miejscu została opisana implementacja bloku wstępnego przetwarzania opisanego w \textbf{sekcji \ref{archfeatureextraction}}.

\subsubsection{Ramkowanie}

Głównym zadaniem bloku wstępnego przetwarzania w implementowanym systemie jest ramkowanie sygnału mowy. Realizowane jest to za pomocą funkcji \textit{frame\_signal}. Funkcja ta przyjmuje jako argument kontener zawierający próbki sygnału \textit{probes} oraz krok przesunięcia kolejnych ramek M. W poniżej implementacji użyta została funkcja \textit{std::copy} umieszczająca w wynikowym wektorze kolejne ramki sygnału.

\begin{lstlisting}[style=lst:cpp, caption=Implementacja funkcji \textit{frame\_signal}\label{lst:frame}]
  template <int N>
  std::vector<std::array<double, N>> frame_signal(const std::vector<double> probes, int M)
  {
    std::vector<std::array<double, N>> speech_frames;
    for(int signal_offset = 0; signal_offset + N < static_cast<int>(probes.size()); signal_offset += M)
    {
      auto frame = std::array<double, N>();
      auto start_pos = probes.cbegin() + signal_offset;
      std::copy(start_pos, start_pos + N, frame.begin());
      speech_frames.push_back(std::move(frame));
    }
    return speech_frames;
  }
\end{lstlisting}

\section{Implementacja ekstrakcji cech MFCC}

W tym miejscu została opisana implementacja bloku ekstrakcji cech opisanego w \textbf{sekcji \ref{archfeatureextraction}}.

Aby utrzymać wektory sparametryzowanego sygnały mowy w postaci współczynników MFCC nazleży wywołać funkcję \textit{mfcc\_extraction} przyjmującą za argumenty częstotliwość próbkowania \textit{samplerate} oraz wektor ramek o długości N z próbkami sygnału mowy \textit{speech\_frames}. Funkcja zwraca wektor tablic długości K cech MFCC:

\begin{lstlisting}[style=lst:cpp, caption=Implementacja funkcji mfcc\_extraction\label{lst:mfcc}]

  std::vector<std::array<double, K>>
  mfcc_extraction( std::vector<std::array<double, N>>&& speech_frames,
                   int samplerate)
  {
    (...)
\end{lstlisting}

    \subsubsection{Obliczanie widma mocy dla ramek oraz nałożenie filtrów mela}

Tak jak opisano w sekcji(\textbf{\ref{mfcc}}) pierwszym etapem ekstrakcji cech w tym przypadku jest obliczenie widma gęstości mocy. Kolejno w pętli wykonuje się operacje nakładania widma (za pomocą obiektu funkcyjnego \textit{hamming\_generator}), obliczenia współczynników mocy DTF za pomocą algorytmu FFT (funkcja \textit{power\_power\_frame}). Kolejnym etapem jest nakładanie filtrów mela za pomocą funkcji \textit{mel\_frame} : 

\begin{lstlisting}[style=lst:cpp, caption=funkcja \textit{mfcc\_extraction} - obliczanie widma gęstości mocy\label{lst:mfccpower}]
    (...)
    std::vector<std::array<double, K>> mel_coefs_speech_frames;
    for(auto &frame: speech_frames)
    {
      dsp_utils::window_frame( frame,
                               dsp_utils::Window_type::hamming_generator);
      dsp_utils::power_fft_frame(frame);
      mel_coefs_speech_frames.push_back(mel_frame(frame, samplerate));
    }
    (...)
\end{lstlisting}

\subsubsection{Nałożenie logarytmu}

W wektorze \textit{mel\_coefs\_speech\_frames} znajdują się na tym etapie funkcje gęstości widma. Kolejnym krokiem jest nałożenie logarytmu dziesiętnego na każdą ramkę. Realizowane jest to za pomocą obiektu funkcyjnego - lambdy: 

\begin{lstlisting}[style=lst:cpp, caption=funkcja \textit{mfcc\_extraction} - obliczanie logarytmu widma gęstości mocy \label{lst:mfcclog}]
    (...)
    for(auto &mel_frame: mel_coefs_speech_frames)
    {
      std::for_each( mel_frame.begin(),
                     mel_frame.end(),
                     [](double &val)
                     {
                       val = std::log10(val); }
                     );
    } 
    (...)
\end{lstlisting}

\subsubsection{Oblicznie cepstrum}

Kolejnym krokiem jest obliczenie cepstrum za pomocy dyskretnej transformaty kosinusowej (DCT). Aby zmniejszyć złożoność obliczeniową najpierw tablicowana jest funkcja kosinus. Dużym usprawnieniem jest jej deklaracja jako \textit{const} tak aby tą operację wykonać już w czasie kompilacji. Stablicowana funkcja znajduje się w tablicy \textit{cos\_table}. DCT realizowana jest za pomocą obiektu funkcyjnego \textit{dct\_frame}.
    
\begin{lstlisting}[style=lst:cpp, caption=funkcja \textit{mfcc\_extraction} - obliczanie cepstrum \label{lst:mfcccepstrum}]
    (...)
    std::array<double, 4*K> cos_table;
    std::generate( cos_table.begin(),
                   cos_table.end(),
                   dsp_utils::cos_dct_gen(4*K));

    std::vector<std::array<double, K>> mfcc;
    for(const auto &mel_frame: mel_coefs_speech_frames)
    {
      mfcc.push_back(dsp_utils::dct_frame(mel_frame, cos_table));
    }
    return mfcc;
  }
\end{lstlisting}

W wyniku wykonanych operacji otrzymany zostaje wektor tablic ze współczynnikami MFCC.

\subsection{Funkcje i obiekty funkcyjne realizujące algorytmy}

W podrozdziale zostały opisane kolejne zaimplementowane w systemie funkcje i obiekty funkcyjne realizujące algorytmy dla zadania obliczenia współczynników MFCC użytych wcześniej w funkcji \textit{mfcc\_extraction}. 
\begin{itemize}
\item{\textbf{Funkcja window\_frame} aplikująca okno czasowe na ramkę:
 \begin{lstlisting}[style=lst:cpp]
void window_frame( std::array<double, N>& fr,
                   const Window_type& win_type);
\end{lstlisting}
}
 - przyjmuje jako argument ramkę sygnału \textit{fr} o rozmiarze N, oraz typ enumerowany \textit{win\_type} określający typ zastosowanego okna. W systemie zastosowano okno Hamminga. Aplikacja okna wykonana jest za pomocą funkcji \textit{std::for\_each} z użyciem obiektu funkcyjnego \textbf{hamming\_generator} z określonym operatorem \textit{operator()} jako:
 \begin{lstlisting}[style=lst:cpp]
void hamming_generator::operator()(double& x) 
{
  x = x*(0.54 - 0.46*gsl_sf_cos(2.0*M_PI*(index++)/(size -1)));
}
\end{lstlisting}

\item{\textbf{Funkcja power\_fft\_frame} obliczająca widmową gęstość mocy dla ramki:
 \begin{lstlisting}[style=lst:cpp]
  void power_fft_frame(std::array<double, N>& fr);
\end{lstlisting}
}
- przyjmuje za argument referencję do pojedynczej ramki sygnału \textit{fr} i modyfikuje ją do postaci współczynników widmowej gęstości mocy. 

Najpierw użyty zostaje algortym fft na modyfikowanej ramce:
\begin{lstlisting}[style=lst:cpp]
    gsl_fft_real_radix2_transform(fr.data(), 1, N);
\end{lstlisting}

Następnie każdy współczynnik DFT zostaje podniesiony do kwadratu.
\begin{lstlisting}[style=lst:cpp]
    std::for_each(fr.begin(),
                  fr.end(),
                  [](double &x)
                  {
                    x = gsl_pow_2(x);
                  });
\end{lstlisting}

A następnie zostają zsumowane wartości rzeczywiste z urojonymi dla kolejnych próbek DFT. Należy zwrócić uwagę, że w wyniku użycia funkcji \textit{gsl\_fft\_real\_radix2\_transform} otrzymano tylko połowę widma z wartościami rzeczywistymi rosnącymi kolejno od 0 do N/2 i wartościami urojonymi z malejącymi indeksami od końca wektora fr. Z tego powodu użyty został iterator rekursywny \textit{crbegin}:

\begin{lstlisting}[style=lst:cpp]
    std::transform(fr.cbegin() +1, fr.cbegin() + (N/2),
                   fr.crbegin(),
                   fr.begin() +1,
                   std::plus<double>());
\end{lstlisting}
W wyniku tych operacji otrzymywany jest wektor współczynników amplitudowego widma mocy.

\item{\textbf{Funkcja mel\_frame} aplikująca filtry melowe:
 \begin{lstlisting}[style=lst:cpp]
  std::array<double, K> mel_frame( const std::array<double, N>& fr,
                                   int samplerate);
\end{lstlisting}
}
- przyjmuje za argument częstotliwość próbkowania \textit{samplerate} oraz ramkę ze współczynnikami reprezentującymi widmową gęstość mocy o długości N. Funkcja zwraca wektor cech o długości K. Dla stworzenia wektora cech odpowiada funkcja \textit{std::generate} oraz obiekt funkcyjny \textit{mel\_frame\_generator}:

 \begin{lstlisting}[style=lst:cpp]
    (...)
    std::array<double, K> mel_frame;
    std::generate(mel_frame.begin(),
                  mel_frame.end(),
                  mel_frame_generator<K, N>(fr, samplerate));
    (...)
 \end{lstlisting}

Operator \textit{operator()} tego obiektu funkcyjnego zdefiniowany jest następująco:

 \begin{lstlisting}[style=lst:cpp]
    double operator()(){
 \end{lstlisting}

Obiekt \textit{mel\_frame\_generator} oblicza i przechowuje wartości indeksów dla początku filtra (\textit{f\_begin}, środku filtra (\textit{f\_center}) oraz końcu filtra (\textit{f\_end}). Do obliczenia wartości szukanego współczynnika wystarczy użyć niezerowych wartości filtru tzn. tylko od indeksu f\_begin do f\_end. Ten zakres jest użyty w dla funkcji \textit{for\_each} w listingu poniżej. Jako funktor użyty jest kolejny obiekt funkcyjny \textit{triangle\_windowed\_sum} inicjowany informacją o ilości próbek dla zbocza rosnącego (f\_center - f\_begin) oraz ilość próbek dla zbocza opadającego (f\_end - f\_begin) używanego filtru trójkątnego. Dla każdego kolejnego współczynnika melowego indeksy położenia filtru trójkątnego są aktualizowane prywatną funkcją \textit{update\_filter\_samples}. Opisana część funkcji operatora to: 
 \begin{lstlisting}[style=lst:cpp]
      auto mel_coef = std::for_each(fram.cbegin() + f_begin,
                      fram.cbegin() + f_end,
                      triangle_windowed_sum(f_center-f_begin,
                                            f_end-f_begin));
      update_filter_samples();
      return mel_coef.acc;
    };
 \end{lstlisting}

 Mnożenie skalarne filtra przez ramkę realizowane jest przez wspomniany obiekt \textit{triangle\_windowed\_sum}. Kolejne próbki (\textit{val}) są przemnażane za pomocą jego przeciążonego operatora \textit{operator()}. Co krok aktualizowana jest wartość filtra - zmienna \textit{win}. Gdy algorytm dojdzie do indeksu środkowego filtru (warunek \textit{cen == i}) zmienia się przyrost wartości filtra na ujemny, co pozwala na zachowanie takiego samego kodu dla obu zboczy filtru. Listing omawianej funkcji znajduje się poniżej:
 \begin{lstlisting}[style=lst:cpp]
    void operator()(const double& val){
      if(cen == i) del = -1.0/static_cast<float>((end-cen));

      acc += win*val;
      win += del;
      i++;
    }
 \end{lstlisting}

\item{\textbf{Obiekt funkcyjny cos\_dct\_gen} generujący funkcję kosinus:
 \begin{lstlisting}[style=lst:cpp]
  struct cos_dct_gen
  {
    cos_dct_gen(int _K): K{_K}, i{0}
    {}

    double operator()(){
      double c = gsl_sf_cos(2*M_PI*static_cast<double>(i)/K);
      ++i;
      return c;
    }
    int K, i;
  };
\end{lstlisting}
}
- korzystając z bibliotecznej funkcji \textit{gsl\_sf\_cos} generowana jest jeden okres funkcji kosinus dla K próbek. Tak wygenerowana tablica jest wykorzystywana dalej w algorytmie obliczania DCT. Warto zauważyć, że ze względu na definicję \textbf{\ref{dct}} potrzebna jest 4 razy większa tablica.


\item{\textbf{Funkcja dct\_frame} obliczająca dyskretną transformatę kosinusową:
 \begin{lstlisting}[style=lst:cpp]
  std::array<double, K>
  dct_frame(const std::array<double, K>& mel_frame,
            const std::array<double, K*4>& cos_table);
\end{lstlisting}
}
- przyjmuje jako argument tablicę współczynników melowych\textit{mel\_frame} o długości K oraz stablicowaną funkcję kosinus o długości 4*K. Funkcja zwraca tablicę współczynników MFCC.

  Funkcja oblicza współczynniki DCT przy pomocy funkcji \textit{std::generate} oraz obiektu funkcyjnego \textit{mfcc\_gen} który generuje kolejne współczynniki DCT. Obiekt inicjalizowany jest referencjami do ramki cech i tablicy okresu funkcji kosinus:
 \begin{lstlisting}[style=lst:cpp]
 (...)
  std::array<double, K> mfcc_frame;
  std::generate(mfcc_frame.begin(),
                mfcc_frame.end(),
                mfcc_gen<K>(mel_frame, cos_table));
  (...)
\end{lstlisting}

Obiekt generuje współczynniki DCT za pomocą przeciążonego operatora \textit{operator()} zgodnie z definicją z rozdziału \textbf{\ref{dct}}:
 \begin{lstlisting}[style=lst:cpp]
    double operator()(){
      double mfcc;
      auto k = 0;
      {// for k=0
        mfcc = (std::sqrt(K/2.0))*(1.0/K)*mel[k];
      }
      for(k=1; k<K; ++k)//exclude mean val
      {
        mfcc+=(std::sqrt(K/2.0))*(2.0/K)*mel[k]*cos[(k*(2*n+1))%(4*K)]; 
      }
      ++n;
      return mfcc; 
    };
 \end{lstlisting}
\end{itemize}

\section{Implementacja rozpoznania wzorca VQ}

W tym miejscu została opisana implementacja bloku rozpoznawania wzorca opisanego w \textbf{sekcji \ref{archpatternmatching}}.

\subsection{Miara dopasowania modelu do zbioru wektorów}

W celu otrzymania wartości miary dopasowania modelu złożnego z C centroid o wymiarze K w przestrzeni wektorów cech należy wywołać funkcję \textit{compute\_distortion}:

\begin{lstlisting}[style=lst:cpp, caption=Implementacja funkcji \textit{compute\_distortion}\label{lst:distortion}]

   template<int C, int K>
   double
   compute_distortion(const std::array<std::array<double, K>, C>& code,
                      const std::vector<std::array<double, K>>&
                                                       acoustic_vectors)
{
  (...)
\end{lstlisting}
Funkcja ta zwraca wartość typu double reprezentującą miarę dopasowania do badanego modelu. Jej argumentami jest książka kodów \textit{code} reprezentująca model mówcy za pomocą C centroid K wymiarowych. Drugim argumentem jest zbiór wektorów cech o tym samym wymiarze K. 

  Implementacja tej funkcji składa się na dwie główne pętle for: pierwsza iteruje po wszystkich wektorach cech \textit{acoustic\_vectors} oraz sumuje wartość odchylenia od modelu dla każdego wektora do zwracanej zmiennej \textit{distortion}. Druga zaś pętla for - wewnętrzna - wyszukuje najbliższej centroidy w celu obliczenia miary dopasowania \textit{distortion}. Opisywana reszta implementacji znajduje się tutaj: 

\begin{lstlisting}[style=lst:cpp, caption=Ciało funkcji \textit{compute\_distortion}\label{lst:distortionbody}]
    (...)
    double distortion = 0;
    for(const auto& acc_vec: acoustic_vectors)
    {
      auto acc_vec_dist = std::numeric_limits<double>::max();
      for(const auto& centroid: code)
      {
        auto distance = vq::dis_eu<K>(centroid, acc_vec);
        if(distance < acc_vec_dist)
          acc_vec_dist = distance;
      }
      distortion += acc_vec_dist;
    }
    return distortion;    
  }
\end{lstlisting}

W powyższym kodzie używa się funkcji \textit{dis\_eu} obliczającej odległość euklidesową pomiędzy dwoma wektorami - \textit{v1} i \textit{v2}. Zaimplementowano funkcję zgodnie z definicją odległości euklidesowej:

\begin{lstlisting}[style=lst:cpp, caption=Funkcja \textit{dis\_eu}\label{lst:eu}]
  template<int K>
  double dis_eu(const std::array<double, K>& v1, const std::array<double, K>& v2)
  {
    double dist =0;

    auto i1 = v1.cbegin();
    auto i2 = v2.cbegin();
    for(;i1 != v1.cend(); ++i1, ++i2)
    {
      dist += std::pow(*i1-*i2, 2); 
    } 

    return std::sqrt(dist);
  }
\end{lstlisting}

\subsection{Modelowania przy pomocy algorytmu LBG}

W celu otrzymania współczynników modelu mówcy - centroid wyprodukowanych przez algorytm LBG (\textbf{sekcja \ref{lgb}}) na podstawie zbioru treningowych wektorów cech należy wywołać funkcję \textit{lbg}:

\begin{lstlisting}[style=lst:cpp, caption=Deklaracja funkcji \textit{lbg}\label{lst:lbg}]
  template<int C, int K>
  std::array<std::array<double, K>, C>
  lbg(const std::vector<std::array<double, K>>& acoustic_vectors);
\end{lstlisting}

W wyniku wywołania funkcji zwracana jest 'książka kodów' o rozmiarze C. Wejściem jest zbiór wektorów cech o długościach K. Argumentem funkcji są wektory cech o nazwie referencji \textit{acoustic\_vectors}.

Implementacja realizuje algorytm znajdujący się na schemacie blokowym(\textbf{\ref{lgbblock}}). Poniżej zostało przedstawione ciało funkcji z opisanymi w komentarzach nazwami kroków algorytmów. Następnie zostanie szczegółowo opisana implementacja każdego kroku z osobna. 

Omawiana funkcja składa się z części inicjalizacyjnej oraz dwóch głównych pętli: zewnętrznej pętli \textit{for} która za każdą iteracją podwaja ilość centroid aż do uzyskania wielkości C długości książki kodów. Wewnętrzna pętla \textit{while} ma za zadanie powtarzać kroki algorytmu LBG aktualizujące położenie centroid do momentu niewielkiej zmiany. Sygnalizowane jest to zmienną \textit{distortion\_diff\_small\_enough}. Stała \textit{eps} jest współczynnikiem przesunięcia położeń dwóch wygenerowanych centroid względem centroidy z której zostały stworzone. Jednocześnie jest też współczynnikiem określającym krok poniżej którego algorytm nie poprawia położeń centroid. Kontenery zadeklarowane w listingu(\textbf{\ref{lst:lbgbody1}}) przechowują inormację o:
\begin{itemize}
\item{\textbf{owner\_ind}} - jest kontenerem przechowującym numer najbliższej centroidy dla każdego wektora cech z \textit{acoustic\_vectors}.
\item{\textbf{owner\_dist}} - jest kontenerem przechowującym odpowiadającą odległość do najbliższej centroidy dla każdego wektora cech.
\item{\textbf{owner\_ind\_count}} - jest kontenerem przechowującym informację o ilości wektorów cech leżących najbliżej dla każdej centroidy.
\item{\textbf{owner\_distortion}} - jest kontenerem przechowującym sumę miar dopasowania wektorów cech przynależących kolejno do każdej centroidy.
\end{itemize}

\begin{lstlisting}[style=lst:cpp, caption=Ciało funkcji \textit{lbg}\label{lst:lbgbody1}]
  template<int c, int k>
  std::array<std::array<double, k>, c>
  lbg(const std::vector<std::array<double, k>>& acoustic_vectors);
  {
    std::array<std::array<double, k>, c> codebook;

    /* inicjalizacja - obliczenie polozenia pierwszej centroidy */ 

    (...)

    const double eps = 0.01; 

    std::vector<int> owner_ind(acoustic_vectors.size(), 0);
    std::vector<double> owner_dist(acoustic_vectors.size(), 0);
    std::vector<int> owner_ind_count(c, 0);
    std::vector<double> owner_distortion(c, std::numeric_limits<double>::max());

    for(auto code_size= 1; code_size < c;)
    {
      /* podwojenie ilosci centroid w ksiazce kodow */

      (...)


      auto distortion_diff_small_enough = false;
      while(!distortion_diff_small_enough)
      {

        /* znalezienie najblizszych centroid dla kazdego wektora cech */

        (...)

        /* aktualizacja polozenia centroid ze wzgledu na przynaleznosc wektrorow cech */

        (...)

        /* obliczenie sumy miary dopasowania dla kazdego wektora cech. */

        (...)

        /* decyzja o kontynuowaniu dostosownia centroid do wektorow cech */

        (...)
      }
    return codebook;
  }
\end{lstlisting}

\subsubsection{Inicjalizacja - pierwsza centroida.}

W tej części funkcji obliczany jest wektor średni - centroida dla wszystkich wektorów cech znajdujących się w \textit{acoustic\_vectors}. Wektor sumy poszczególnych wymiarów jest utworzony jako pierwsza centroida w książce kodów \textit{code} za pomocą funkcji \textit{std::transform}, a następnie w drugiej pętli for liczona jest średnia poprzez podzielenie sum przez ilość wektorów cech.

\begin{lstlisting}[style=lst:cpp, caption=Funkcja \textit{lbg}\label{lst:lbgbody2} - obliczenie pierwszej centroidy]
    std::fill(code.at(0).begin(), code.at(0).end(), 0);
    for(auto&& a_vec: acoustic_vectors)
    {
      std::transform(a_vec.cbegin(), a_vec.cend(),
                     code.at(0).cbegin(), code.at(0).begin(),
                     [](double x, double acc){
                       return x+acc;
                     }); 
    }

    for(auto&& x: code.at(0))
    {
      x /= acoustic_vectors.size();
    }
\end{lstlisting}
\subsubsection{Podwojenie ilości centroid.}
Poniższy fragment kodu kopiuje każdą centroidę modyfikując każdy wymiar o współczynnik \textit{1+esp} oraz dodaje ją do książki kodów \textit{code}. Pozostałe centroidy są modyfikowane o współczynnik \textit{1-esp} w drugiej pętli for.
\begin{lstlisting}[style=lst:cpp, caption=Funkcja \textit{lbg}\label{lst:lbgbody3} - podwojenie ilości centroid]
    (...)

      for(auto j = 0; j < code_size; ++j)
      {
        std::transform(code.at(j).cbegin(),
                       code.at(j).cend(),
                       code.at(j+code_size).begin(),
                       [eps](double x){
                         return x*(1.0+eps); 
                       });
        for(auto && x: code.at(j))
        {
          x *= 1.0-eps;
        }
      }
      code_size *= 2;

    (...)
\end{lstlisting}
\subsubsection{Aktualizacja położeń centroid.}
Poniższy kod realizuje funkcjonalność detekcji najbliżej centroidy z książki kodów \textit{code} oraz obliczenia wartości miary dopasowania dla każdego wektora cech. Główna pętla for iterująca po każdym z wektorów cech aktualizuje kolejne wartości kontenerów \textit{owner\_ind}, \textit{owner\_dist} i \textit{owner\_ind\_count}. Znalezienie najbliższej centroidy odbywa się poprzez analizę odległości euklidesowej do każdej centroidy przy pomocy funkcji \textit{dis\_eu}.
\begin{lstlisting}[style=lst:cpp, caption=Funkcja \textit{lbg}\label{lst:lbgbody4} - znalezienie najbliższej centroidy]
      (...)

        std::fill(owner_ind_count.begin(), owner_ind_count.end(), 0);
        //find owner indexes
        auto a_vec_index =0;
        for(auto a_vec_iter = acoustic_vectors.cbegin();
            a_vec_iter != acoustic_vectors.cend();
            ++a_vec_iter, ++a_vec_index)
        {
          std::vector<double> distances;
          for(auto iter = code.cbegin();
              iter != code.cbegin() +code_size;
              ++iter)
          {
            distances.push_back(vq::dis_eu<K>(*a_vec_iter, *iter));
          } 
          auto min_dist = std::min_element(distances.begin(),
                                           distances.end());
          auto cen_index = std::distance(std::begin(distances), min_dist);
          owner_ind.at(a_vec_index) = cen_index;
          owner_dist.at(a_vec_index) = distances.at(cen_index);
          ++owner_ind_count.at(cen_index);
        }

     (...)
\end{lstlisting}

Kolejnym krokiem jest aktualizacja centroid poprzez obliczenie średniego wektora dla każdej z nich przez uwzględnienie tylko tych wektorów cech które należą do segmentu tej centroidy. Realizowane jest to w linii 16 listingu \textbf{\ref{lst:lbgbody5}} za pomocą funkcji \textit{std::transform} oraz kolejnej pętli for w której wykonywane jest dzielenie.
\begin{lstlisting}[style=lst:cpp, caption=Funkcja \textit{lbg}\label{lst:lbgbody5} - obliczenie ogólnego dopasowania]
      (...)

        for(auto cen_iter = code.begin();
            cen_iter !=  code.begin() +code_size;
            ++cen_iter)
        {
          std::fill(cen_iter->begin(), cen_iter->end(), 0);
        }
        a_vec_index =0;
        for(auto a_vec_iter = acoustic_vectors.cbegin();
            a_vec_iter != acoustic_vectors.cend();
            ++a_vec_iter, ++a_vec_index)
        {
          auto owner_centroid_begin =
                              code.at(owner_ind.at(a_vec_index)).begin();
          std::transform(a_vec_iter->cbegin(), a_vec_iter->cend(),
                         owner_centroid_begin, owner_centroid_begin,
                        [](double x, double y){
                          return x+y; 
                        });
        }
        auto cen_index = 0;
        for(auto cen_iter = code.begin();
            cen_iter !=  code.begin() +code_size;
            ++cen_iter, ++cen_index)
        {
          for(auto && c: *cen_iter)
          {
            auto count = owner_ind_count.at(cen_index);
            c /= count!=0?count:1;
          }
        }

      (...)
\end{lstlisting}
\subsubsection{Suma miary dopasowania i decyzja o kontynuowaniu.}
Ostatnią operacją wykonywaną w omawianej funkcji jest obliczenie sumy miar dopasowania które zostały już obliczonych we wcześniejszych etapach i znajdujących się w kontenerze \textit{owner\_dist}. Sumowanie jest realizowane w pierwszej pętli for w listingu \textbf{\ref{lst:lbgbody6}} co w rezultacie daje wektor \textit{distortion} posiadający sumę miar dopasowania dla każdej centroidy. Kolejna pętla for implementuje logikę decyzji o tym czy kontynuować procedurę aktualizacji centroid. Wykonywane jest to poprzez sprawdzenie czy przesunięcie którejkolwiek centroidy nie było większe niż współczynnik \textit{esp}. Jeżeli ten warunek jest prawdą wtedy procedura odszukiwania najbliższych centroid i aktualizacji tychże jest uruchamiana ponownie poprzez odpowiednie ustalenie stanu zmiennej \textit{distortion\_diff\_small\_enough}.
\begin{lstlisting}[style=lst:cpp, caption=Funkcja \textit{lbg}\label{lst:lbgbody6} - decyzja o kontynuowaniu]
      (...)

        std::vector<double> distortion(owner_distortion.size(),0);
        cen_index = 0;
        for(auto i=0ul; i < acoustic_vectors.size(); ++i)
        {
          auto owner_index = owner_ind.at(i);
          distortion.at(owner_index) += owner_dist.at(i);
        }
        for(auto i=0; i < code_size; ++i)
        {
          distortion_diff_small_enough = true;
          auto d_coef =
            (owner_distortion.at(i)-distortion.at(i))/distortion.at(i); 
          if(d_coef > eps)
          {
            distortion_diff_small_enough = false;
            owner_distortion = std::move(distortion);
            break;
          }
        }

      (...)
\end{lstlisting}

\section{Implementacja algorytmu decyzji przy pomocy ustalonego progu $\theta$}

Została zaimplementowana najprostsza metoda dla deterministycznego modelowania, opisana w sekcji(\textbf{\ref{fixed}}).

\begin{lstlisting}[style=lst:cpp, caption=Funkcja \textit{make\_decision}\label{lst:dec} - decyzja o kontynuowaniu]
  inline bool make_decision(double distortion, double theta)
  {
    return distortion < theta;
  }

\end{lstlisting}

W tym miejscu została opisana implementacja bloku podejmowania decyzji opisanego w \textbf{sekcji \ref{archpatternmatching}}.
