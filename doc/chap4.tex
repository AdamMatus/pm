\chapter{Testy i podsumowanie}

\section{Test ERR dla 11 mówców w bazie.}

Zaimplementowany system weryfikacji mówcy został poddany statycznemu testowi na współczynnik błędu EER (\textit{ang. Equal Error Rate}). Błąd ten jest zdefiniowany dla takiego systemu z progiem $\theta$, dla którego błąd typu fałszywa akceptacja FAR - (\textit{ang. False Acceptance Rate}) jest równy błędowi typu fałszywa odmowa - FRR (\textit{ang. False Rejection Ratio}).

\label{err}

Błąd FAR jest zdefiniowany następująco:
  \begin{equation}
  FAR = \frac{n_{fa}}{n_r} \cdot 100\%
  \end{equation}
  gdzie $n_{fa}$ oznacza ilość przypadków zaakceptowania mówcy, który nie jest mówcą weryfikowanym, $n_{r}$ oznacza całkowitą ilość prób weryfikacji takiego mówcy.

Błąd FRR jest zdefiniowany następująco:
  \begin{equation}
  FRR = \frac{n_{fr}}{n_a} \cdot 100\%
  \end{equation}
  gdzie $n_{fr}$ oznacza ilość przypadków odrzucenia mówcy, będącego mówcą weryfikowanym. $n_{a}$ oznacza całkowitą ilość prób weryfikacji takiego mówcy.


W testowanym systemie zastosowano K=30 współczynników MFCC, ramki sygnału o długości N=256 oraz C=16 centroid dla modelu mówcy przy użyciu techniki VQ. Test składał się z przeanalizowania wpływu zależności stałego współczynnika $\theta$ (rozdział {\ref{fixed}}) z szerokiego zakresu wartości na wielkość błędów FAR i FRR w systemie weryfikacji mówcy. W teście użyto 16 modeli mówców i 16 nagrań testowych angielskiego słowa 'zero' od tych samych mówców w celu oszacowania błędów FRR. Daje to 256 testów weryfikacyjnych dla każdej wartości współczynnika $\theta$. Dla oszacowania błędu FAR użyto 3 modeli mówców pochodzących z tej samej książki kodów wraz z 32 nagraniami pochodzącymi od tych 3 mówców. Wyniki testu przedstawione są na rysunku (\textbf{\ref{fig:err}}). Wynik ERR wyniósł 6\%.

Ze względu na małą ilość danych treningowych oraz danych testowych (szczególnie w przypadku szacowania błędu FAR) otrzymane wyniki mogą różnić się w pewnym stopniu od statystycznie istotnej oceny modelu.


\begin{figure}[!h]
  \centering
    \input{err.pgf}
    \caption{\label{fig:err} Wykres zależności błędu fałszywych akceptacji i odrzuceń.}
\end{figure}

\section{Podsumowanie}

Powstały w niniejszej pracy system weryfikacji mówcy osiągnął dobry wynik błędu ERR równy 6\% w teście zawierającym ograniczony materiał nagrań. Każde nagranie zostało zarejestrowane w podobnych warunkach z podobnymi zakłóceniami tła oraz zawierało tę samą treść (wyświetlane hasło). Z tych powodów aby otrzymać bardziej wiarygodne wyniki efektywności otrzymanego systemu należy wykonać bardziej rozległe testy, najlepiej przy użyciu dużych baz wypowiedzi mówców powstałych w tym
właśnie celu, takie jak baza NIST (\textit{National Institute of Standards and Technology}) albo ELRA (European Language Resources Association). Implementacja otrzymanego przykładowego systemu składa się tylko z niezbędnych elementów potrzebnych do jego uruchomienia. Jednak przedstawiona w rozdziale 2 architektura systemu pozwala na szeroką ingerencję, jeżeli chodzi o dodawanie do systemu nowych technik poprawiających efektywność systemu. W uruchomionym na potrzeby tej pracy
systemie pożądanym ulepszeniem jest wprowadzenie technik detekcji mówcy (VAD) tak, aby puste ramki cech MFCC (pomiędzy słowami) nie wpływały na efekt procesu tworzenia modelu mówcy (ze względu na faworyzowanie modeli z centroidami bliżej punktu zerowego). Tak jak wspomniano na wstępie pracy, warto dostarczyć do systemu więcej materiału trenującego (np. parokrotne powtórzenia wszystkich cyfr dla każdego mówcy w przypadku systemu z wyświetlanym hasłem). Kolejnym ulepszeniem systemu
mogłoby być w zastosowaniach z przetwarzaniem sygnału mowy w czasie rzeczywistym uwzględnienie podsystemu \textit{speaker pruning} (rozdział {\ref{speakerprunning}}) może to okazać się przydatne dla zwiększenia szybkości pracy systemu dla weryfikacji z modelami kohort.

Autor uważa, że zaproponowana architektura i sposób implementacji w języku C++ może być bazą do stworzenia obszernego narzędzia dla prac związanych z konstruowaniem systemów weryfikacji mówcy dla systemów wbudowanych w czasie rzeczywistym. Wydaje się, że jest możliwa efektywna i jasna implementacja w proponowanym środowisku większości technik weryfikacji mówcy omawianych w części teoretycznej niniejszej pracy.
